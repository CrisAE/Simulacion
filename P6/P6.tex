\documentclass{article}
\usepackage[spanish]{babel}
\usepackage[utf8]{inputenc}
\usepackage[T1]{fontenc}
\usepackage{graphicx}
\usepackage{listings}
\usepackage{amsmath}
\usepackage[numbers,sort&compress]{natbib}


\title{\bf Práctica 6: sistema multiagente}
\date{\today}
\author{C. A. Estrada}

\begin{document}

\maketitle

\section{Objetivo}
Estudiar el efecto estadístico de de la implementación de una vacuna con probabilidad $p_v$ \cite{dra}, con valor de cero a uno en pasos de 0.1, en el porcentaje máximo de infectados y el momento (iteración) en el cual se alcanza ese máximo. 

\section{Metodología}
Se emplea el paquete estadístico R versión 4.0.2 \cite{R} para la generación del código, tomando como base lo ya previamente reportado para simular una epidemia \cite{dra}, en donde los agentes podrán estar en uno de tres estados: susceptibles (S), infectados (I) o recuperados (R). La implementación de una vacuna con probabilidad $p_v$ a los agentes al inicio de experimento propicia que se encuentren desde el comienzo en estado R; la variación de $p_v$ se ubica entre cero y uno en pasos de 0.1, y para cada experimento se realizan diez repeticiones.

Se extrae el valor máximo de infectados obtenido en cada experimento, así como la iteración (tiempo) en que se presenta este valor y se reporta aquella en que se alcanza el valor máximo de infectados entre las diez repeticiones.
\begin{lstlisting}[language=R]
    mayorinf=0  
    pasosmax=0 
    for (tiempo in 1:tmax){ 
      infectados <- dim(agentes[agentes$estado == "I",])[1]
      epidemia <- c(epidemia, infectados)
      if (infectados == 0) {
        pasosmax=tiempo
        break 
      } 
      if(max(epidemia)>mayorinf){
        mayorinf=max(epidemia)
        pasosmax=tiempo
      }
\end{lstlisting}

Para el cálculo del porcentaje máximo de infectados se emplea el valor máximo de infectados generado por cada experimento.
\begin{lstlisting}[language=R]
porcinf = (mayorinf / n) * 100
\end{lstlisting}

Finalmente, con el valor de los porcentajes obtenidos para cada probabilidad se genera un gráfico y se realiza la prueba estadística ANOVA para observar el efecto de la probabilidad de vacunación sobre el porcentaje de infectados.

\section{Resultados y discusión}
En el cuadro \ref{tabla} se presentan los porcentajes del máximo de agentes infectados obtenidos para cada probabilidad de vacunación, así como el tiempo (iteración) en que se presentan; se observa a simple vista que al aumentar la probabilidad inicial de la vacuna, el procentaje máximo de infectados disminuye, hasta llegar a cero para la probabilidad de 1.0, mientras que el tiempo en que se presentan los picos de infección no presenta ningún patrón respecto al aumento de la probabilidad o la cantidad de infectados. Para la máxima probabilidad de vacunación, de 1.0, no se presenta ningún infectado al inicio del experimento, por lo que el tiempo y el porcentaje son iguales a cero.
\begin{table}[h]
\begin{center}
\caption{Porcentajes máximos de agentes infectados obtenidos por cada probabilidad inicial de la vacuna y el tiempo (iteración) en que se presentan.}
\label{tabla}
\begin{tabular}{c c c}
\hline
\textbf{Probabilidad}&\textbf{Porcentaje}&\textbf{Tiempo}\\
\hline
0.0&70\%&36 \\
0.1&76\%&33\\
0.2&60\%&45\\
0.3&42\%&96\\
0.4&40\%&56\\
0.5&26\%&53\\
0.6&24\%&54\\
0.7&20\%&70\\
0.8&8\%&85\\
0.9&4\%&71\\
1.0&0\%&0\\
\hline
\end{tabular}
\end{center}
\end{table}

En la figura \ref{probs} se presentan los porcentajes del máximo de infectados tomando en cuenta las repeticiones de cada experimento respecto a cada probabilidad de vacunación, y se observa de manera gráfica la disminución gradual del número de infectados conforme se aumenta la probabilidad inicial de la vacuna. La prueba ANOVA realizada (cuadro \ref{anova}) verifica que existen diferencias significativas entre los porcentajes del máximo de infectados con respecto a la probabilidad inicial de vacunación en los experimentos realizados, ya que se obtiene una significancia menor a 0.05, rechazando la hipótesis nula.
\begin{figure}[ptb]
\begin{center}
\includegraphics[width=\linewidth]{p6.png}
\end{center}
\caption{Porcentajes del máximo de agentes infectados obtenidos para cada probabilidad inicial de vacunación.\label{probs}}
\end{figure}

\begin{table}[h]
\begin{center}
\caption{Comparación de los porcentajes del máximo de infectados con respecto a la probabilidad inicial de vacunación.}
\label{anova}
\begin{tabular}{c c c c c c}
\hline
 &\textbf{GL}&\textbf{Suma Cuad.}&\textbf{Media Cuad.}&\textbf{Valor F}&\textbf{Pr(>F)}\\
\hline
Probabilidad&1&38055&38055&203.4&<2e-16\\
Residuales&108&20210&187\\
\hline
\end{tabular}
\end{center}
\end{table}

\section{Conclusión}
El aumento de la probabilidad inicial de vacunación genera una disminución en el porcentaje máximo de agentes infectados en la simulación del sistema multiagente de una epidemia.

\bibliography{P6}
\bibliographystyle{unsrtnat}

\end{document}

