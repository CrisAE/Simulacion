\documentclass[a4paper,twocolumn,12pt]{article}
\usepackage[spanish]{babel}
\usepackage[utf8]{inputenc}
\usepackage[T1]{fontenc}
\usepackage{graphicx}
\usepackage{subfigure}
\usepackage{listings}
\lstset{language=R, breaklines=true}
\usepackage{amsmath}
\usepackage[numbers,sort&compress]{natbib}
\usepackage[top=25mm, bottom=20mm, left=1.5cm, right=1.5cm]{geometry}

\title{\bf Modelado de grano grueso de la proteína triptavidina y su comparación con la estructura cristalográfica experimental}
\date{\today}
\author{C. A. Estrada}

\begin{document}

\twocolumn[
\begin{@twocolumnfalse}
\maketitle
\vspace*{-1cm}
\begin{center}\rule{0.9\textwidth}{0.1mm} \end{center}
\begin{abstract}
\normalsize Aquí se va a escribir el resumen del trabajo, al finalizarlo.\\ \\
Palabras clave: Aquí van.
\begin{center}\rule{0.9\textwidth}{0.1mm} \end{center}
\vspace*{0.5cm}
\end{abstract}
\end{@twocolumnfalse}
]

\section{Introducción}
Primer párrafo: qué es la triptavidina y su importancia en la nanobiotecnología. 

Siguientes párrafos: decribir brevemente (a modo de antecedentes y trabajos relacionados) qué es el modelado de proteínas por coarse-grained y su importancia, así como algunos de los modelos propuestos por dicho método. Descripción del modelo MARTINI que se va a utilizar.

Último párrafo: descripción breve de la utilidad del modelado de la triptavidina en la nanotecnología, así como de lo estudiado en el presente trabajo.


\section{Metodología}
Se describen los parámetros utilizados para la implementación de la simulación, el modelo propuesto y se cita apropiadamente el software y herramientas \cite{R} empleados para ello.


\section{Resultados y discusión}
Agregar la proteína generada por simulación, cálculo de parámetros de calidad (pendiente de revisar cómo hacerlo).

Comparación y análisis estadístico contra la proteína real (los datos experimentales obtenidos en PDB).

¿Agregar análisis de desempeño computacional durante la simulación? 


\section{Conclusiones}


\subsection*{Agradecimientos}


\bibliography{referencias}
\bibliographystyle{unsrt}
\end{document}